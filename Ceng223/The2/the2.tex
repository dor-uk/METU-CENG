\documentclass[12pt]{article}
\usepackage[utf8]{inputenc}
\usepackage{float}
\usepackage{amsmath}
\usepackage{amssymb}
\usepackage[shortlabels]{enumitem}

\usepackage[hmargin=3cm,vmargin=6.0cm]{geometry}
%\topmargin=0cm
\topmargin=-2cm
\addtolength{\textheight}{6.5cm}
\addtolength{\textwidth}{2.0cm}
%\setlength{\leftmargin}{-5cm}
\setlength{\oddsidemargin}{0.0cm}
\setlength{\evensidemargin}{0.0cm}

%misc libraries goes here

\begin{document}

\section*{Student Information } 
%Write your full name and id number between the colon and newline
%Put one empty space character after colon and before newline
Full Name : Doruk Berke Yurtsızoğlu \\
Id Number : 2522225 \\

% Write your answers below the section tags
\section*{Answer 1}

\paragraph{a) $f_1: \mathbb{R} \rightarrow \mathbb{R} \ , f(x) = x^2$}
.\\
\\
Is it surjective ? \\
Let's assume f(x) = y and y = -1\\
Also we know that $f(x) = x^2$ and $x^2 \geq 0$ (scope of f(x) doen't contains negative real number)\\
In conclusion, $y \in \mathbb{R}$ but not in the scope of f(x).\\
Answer: Not surjective\\ 
\\
Is it injective ? \\
For a function to be injective, it has to fulfill one condition;\\
$Condition: \ f(x) = f(y)\ then\ x = y $ \\
Let's assume $x = 5$ and $y = -5$\\
Then both f(x) and f(y) are equal to 25. The condition is not fulfilled.\\
Answer: Not injective\\

			 
 % For f_1
\paragraph{b) $f_2: \mathbb{R^{+}} \rightarrow \mathbb{R} \ , f(x) = x^2$}
 .\\
\\
Is it surjective ? \\
Let's assume f(x) = y and y = -1\\
Also we know that $f(x) = x^2$ and $x^2 \geq 0$ (scope of f(x) doen't contains negative real number)\\
In conclusion, $y \in \mathbb{R}$ but not in the scope of f(x).\\
Answer: Not surjective\\ 
\\
Is it injective ? \\
For a function to be injective, it has to fulfill one condition;\\
$Condition: \ f(x) = f(y)\ then\ x = y $ \\
Let's assume $x,y \in \mathbb{R^{+}}$\\
In order to get the same results from f(x) and f(y), x and y must be equal.\\
The condition is fulfilled.\\
Answer: Injective\\


% For f_2
\paragraph{c)$f_3: \mathbb{R} \rightarrow \mathbb{R^{+}} \ , f(x) = x^2$}
 .\\
\\
Is it surjective ? \\
Let's assume $x \in \mathbb{R}, y \in \mathbb{R^{+}}$\\
Also we know that scope of f(x) is $\mathbb{R^{+}}$, since $x^2 \geq 0$ \\
In conclusion, we can say that for all y values in the scope there exists a x value.\\
Answer: Surjective\\ 
\\
Is it injective ? \\
For a function to be injective, it has to fulfill one condition;\\
$Condition: \ f(x) = f(y)\ then\ x = y $ \\
Let's assume $x = 5$ and $y = -5$\\
Then both f(x) and f(y) are equal to 25. The condition is not fulfilled.\\
Answer: Not injective\\



\paragraph{d)$f_4: \mathbb{R^{+}} \rightarrow \mathbb{R^{+}} \ , f(x) = x^2$}
 .\\
\\
Is it surjective ? \\
Let's assume $x \in \mathbb{R^{+}}, y \in \mathbb{R^{+}}$\\
Also we know that scope of f(x) is $\mathbb{R^{+}}$, since $x^2 \geq 0$ \\
In conclusion, we can say that for all y values in the scope there exists a x value.\\
Answer: Surjective\\ 
\\
Is it injective ? \\
For a function to be injective, it has to fulfill one condition;\\
$Condition: \ f(x) = f(y)\ then\ x = y $ \\
Let's assume $x,y \in \mathbb{R^{+}}$\\
In order to get the same results from f(x) and f(y), x and y must be equal.\\
The condition is fulfilled.\\
Answer: Injective


\section*{Answer 2}
\paragraph{a)}
Let f: $A \subset \mathbb{R}^n \rightarrow \mathbb{R}^m$ and $x_0 \in A.$
For each $\epsilon > 0$, there exists some $\delta > 0$ such that for any $x \in A$ with $\lVert x - x_0\rVert < \delta$ and $\lVert f(x) - f(x_0)\rVert < \epsilon.$ (Formal definition of limit)\\
\\
Take any function $f: \mathbb{Z} \rightarrow \mathbb{R}.$ Pick some $x_0 \in \mathbb{Z}$ and let $\epsilon > 0.$ Then we need to come up with a $\delta$ to make the rest of the statement true. Let's say $\delta = 1/2$ for clarity.\\
Let $x \in \mathbb{Z}$ and suppose $\lVert x - x_0\rVert < (\delta = 1/2).$ Since the only integer within the radius (1/2) of $x_0$ is itself. Then we must have $x = x_0.$ Thus, $f(x) = f(x_0)$ and $\lVert f(x) - f(x_0)\rVert = 0$ which is certainly less than $\epsilon.$
This shows that $f$ is continuous at $x_0.$ Since $x_0$ is arbitrary, we can conclude that $f$ is continuous everywhere on its domain.




\paragraph{b)}
Show that a necessary and sufficient condition for a function $f: \mathbb{R} \rightarrow \mathbb{Z}$ to be continuous is that $f$ is a constant function.\\
\\
We are going to do this by using contradiction. Let's assume $f$ is not constant. Then, we have distinct integers $x\ and\ y$ in the scope of $f$. So, we understand that there are some real numbers $a\ and\ b$ such that $f(x) = a$ and $f(y) = b.$\\
Now we can say that there exists a real number $t$ (*not an integer), which is between $a\ and\ b$ by using "Intermediate Value Theorem".\\
We want $f$ to be continuous; so there must be a real number $z$ between $x\ and\ y$, such that $f(z) = t.$ However, this contradicts the assumption that f takes only integer values. Hence f must be a constant.\\
\\
Now, let's assume $f: A \subset \mathbb{R} \rightarrow \mathbb{Z}$ \\
Since $f$ is a constant function, for every $f(x)$ and $f(x_0)$ we get $f(x) = f(x_0)$ which satisfies the conditions of formal definition of limit. Thus, because the formal definition of limit (the statement in the question) is true for every $f: A \subset \mathbb{R} \rightarrow \mathbb{Z}$, we can say $f$ is continuous by the theorem.\\
\\
In the end, we can conclude that if $f: A \subset \mathbb{R} \rightarrow \mathbb{Z}$ is a constant function, then it is also continuous.\\
*Intermediate Value Theorem: For any function $f$ that is continuous over an interval $[a,b],$ the function will take any value between $f(a)$ and $f(b)$ over the interval.



\section*{Answer 3}
\paragraph{a)}
Show that a finite Cartesian product of countable sets, i.e. $X_n = A_1 \times A_2 \times . . . \times A_n$ for all $n \geq 2$, is countable.\\
\\
Let's define an integer k such that $1 \geq k < n$, then we get $X_k = A_1 \times A_2 \times . . . \times A_k$\\
Let's use induction;\\
If k = 1, $X_1 = A_1$. (Obviously this is countable)\\
After this, we will assume that $X_k$ is countable where $1 < k < n$, so we can write $X_{k+1}$ as: $X_{k+1} = (A_1 \times A_2 \times A_3 ... \times A_k) \times A_{k+1}$ and we know $X_k = A_1 \times A_2 \times . . . \times A_k$ is countable.\\ 
Then, we can write $X_{k+1}$ as $X_k \times A_{k+1}$ $(X_k\ and\ A_{k+1}\ are\ countable)$. Hence $X_{k+1}$ is also countable. ($\mathbb{Z^{+}}\ and\ \mathbb{Z^{+}}\times \mathbb{Z^{+}}$ have the same cardinality.) \\






\paragraph{b)}
Show that an infinite countable product of the set $X = \{0, 1\}$ with itself is uncountable.\\
Statement: Let $X = \{0, 1\}$, then set X that subset of $\mathbb{Z^{+}}$ is uncountable.\\
\\
Proof: (We are going to use Cantor's argument of diagonalization)\\
Any function g: $\mathbb{Z^{+}} \rightarrow X$, we must show that g is not surjective in order to say the set is uncountable.\\
Let g(n) = $(x_{n1}, x_{n2}, x_{n3}... x_{ni}...)$ and $x_{ni} \in \{0,1\},\ i \in \mathbb{Z^{+}}$\\
Then define an element A = $(A_1, A_2, A_3... A_n)$ and $A \in X$ such that\\
\\
$A_n = 0\ if \ x_{nn} = 1$\\
$A_n = 1\ if \ x_{nn} = 0$\\
\\
Such defined A is not mapped by g by the definition of A. A differs from each $x \in X$ but clearly $A \in X$ as well.
So it is uncountable.\\ 





\section*{Answer 4}
\paragraph{a)} % Compare your first and second functions
$L = \lim_{n \rightarrow \infty} (\dfrac{2^n}{n^{50}})$\\
Apply L'Hospital's Rule\\
$L = \lim_{n \rightarrow \infty} (2^n) \cdot (\dfrac{\ln^{50}{2}}{50!})$
$= (\dfrac{\ln^{50}{2}}{50!}) \cdot \infty = \infty$\\
$O(2^n) > O(n^{50})$


\paragraph{b)} % Compare your second and third functions
$L = \lim_{n \rightarrow \infty} (\dfrac{2^n}{(\log{n})^2})$
$= \lim_{n \rightarrow \infty} (\dfrac{2^n}{(2*\ln{n})})$
$= \dfrac{1}{2} \cdot \lim_{n \rightarrow \infty} (\dfrac{2^n}{(\ln{n})})$\\
Apply L'Hospital's Rule\\
$L = \dfrac{1}{2} \cdot \lim_{n \rightarrow \infty} (\dfrac{2^n * \ln{2}}{ \dfrac{1}{n}})$
$= \dfrac{1}{2} \cdot \ln{2} \cdot \lim_{n \rightarrow \infty} {2^n} \cdot \lim_{n \rightarrow \infty} {n}$
$= \dfrac{1}{2} \cdot \ln{2}\cdot \infty \cdot \infty = \infty$\\
$O(2^n) > O(\log{n})^2)$


\paragraph{c)}
$L = \lim_{n \rightarrow \infty} (\dfrac{2^n}{((\log{n})\cdot n^{1/2})})$
$= \infty$\\
$O(2^n) > O((\log{n})\cdot n^{1/2})$
(* We know this from the previous part.)

\paragraph{d)}
Now we need to compare $2^n$ with $5^n$, but we already know that $O(5^n) > O(2^n).$ So we can skip this comparison and look at another one.\\
$L = \lim_{n \rightarrow \infty} (\dfrac{5^n}{(n!)^2})$
$= \lim_{n \rightarrow \infty} e^{\ln{\dfrac{(5^n)}{(n!)^2}}}$
$= e^{\lim_{n \rightarrow \infty} \ln{\dfrac{(5^n)}{(n!)^2}}}$
$= e^{- \infty}$
$= 0$\\
$O(5^n) < O((n!)^2)$
\paragraph{e)}
$L = \lim_{n \rightarrow \infty} (\dfrac{n^{51} + n^{49}}{n^{50}})$
$=  \lim_{n \rightarrow \infty} (\dfrac{n^{2} + 1}{n})$
$= \lim_{n \rightarrow \infty} (n(\dfrac{1 + \dfrac{1}{n^2}}{1}))$
$= \infty \cdot 1 = \infty$\\
$O(n^{51} + n^{49}) > O(n^{50})$


\paragraph{f)}
$L = \lim_{n \rightarrow \infty} (\dfrac{\log{n}^2}{(\log{n})\cdot n^{1/2}})$
$= \lim_{n \rightarrow \infty}  (\dfrac{\log{n}}{n^{1/2}})$\\
Apply L'Hospital's Rule\\
$L = \lim_{n \rightarrow \infty} \dfrac{2}{n^{1/2}}$
$= \dfrac{2}{\infty} = 0$\\
$O(\log{n}^2) < O((\log{n})\cdot n^{1/2})$\\
\\
We have everything we need to know to arrange the functions.\\
The order will be: $\log{n}^2 < (\log{n})\cdot n^{1/2} < n^{50} < n^{51} + n^{49} < 2^n < 5^n < (n!)^2$


\section*{Answer 5}
\paragraph{a)}
Let's say a = 94 and b = 134\\
\\
1. $b > a$ so assign (b - a) to b: b = 134 - 94 = 40\\
2. $a > b$ so assign (a - b) to a: a = 94 - 40 = 54\\
3. $a > b$ so assign (a - b) to a: a = 54 - 40 = 14\\
4. $b > a$ so assign (b - a) to b: b = 40 - 14 = 26\\
5. $b > a$ so assign (b - a) to b: b = 26 - 14 = 12\\
6. $a > b$ so assign (a - b) to a: a = 14 - 12 = 2\\
7. $b > a$ so assign (b - a) to b: b = 12 - 2 = 10\\
8. $b > a$ so assign (b - a) to b: b = 10 - 2 = 8\\
9. $b > a$ so assign (b - a) to b: b = 8 - 2 = 6\\
10. $b > a$ so assign (b - a) to b: b = 6 - 2 = 4\\
11. $b > a$ so assign (b - a) to b: b = 4 - 2 = 2\\
12. $b = a = 2$ so the gcd of 134 and 94 is 2



\paragraph{b)}
Statement: every integer greater than 5 is the sum of three primes\\
\\
{First, we are going to assume Goldbach's Conjecture is valid.}
{After that, we will use it to reach the statement above.}\\
{Moreover, we are going to assume the statement is valid.}
{Then, we will use it to reach to Goldbach's Conjecture.}
{If we can achieve this, we can finally say that}
{Goldbach's Conjecture and the statement are}
{are equivalent.}\\
\\
Let's say we have a variable x. $(x \in \mathbb{Z})$\\
$Assume\ x > 5$\\
\\
There are two cases we should inspect.\\
\\
Case 1: (x is even)\\
If x is even we can say that x = 2n $(n \in \mathbb{Z}\ and\ n \geq 3)$. \\
Then, x - 2 = 2n -2 which is equal to 2(n - 1) ,so x - 2 is also even. \\
Now we are going to apply Goldbach's Conjecture to (x - 2).\\
We know that (x - 2) is even ,so we can write (x - 2) as $(p_1 + p_2)$  \\
$x - 2 = p_1 + p_2$\\
Add 2 to the both sides of the equation.\\
$x = p_1 + p_2 + 2$\\
$p_1, p_2\ and\ 2$ are prime numbers, so x is the sum of three prime numbers.\\
\\
Case 1: (x is odd)\\
If x is odd we can say that x = 2n + 1 $(n \in \mathbb{Z}\ and\ n \geq 3)$. \\
Then, x - 3 = 2n -2 which is equal to 2(n - 1) ,so x - 3 is also even. \\
Now we are going to apply Goldbach's Conjecture to (x - 3).\\
We know that (x - 3) is even ,so we can write (x - 3) as $(p_1 + p_2)$  \\
$x - 3 = p_1 + p_2$\\
Add 3 to the both sides of the equation.\\
$x = p_1 + p_2 + 3$\\
$p_1, p_2\ and\ 3$ are prime numbers, so x is the sum of three prime numbers.\\
\\
Now we are going to prove that the opposite way is also holds.\\
\\
Let's say we have a variable x. $(x \in \mathbb{Z})$\\
$Assume\ x > 2\ and \ x\ is\ even$\\
\\
x = 2n and x + 2 = 2n + 2\\
x + 2 = 2(n + 1), so x + 2 is also even and $x + 2 > 5$\\ 
Thus, x + 2 = $ p_1 + p_2 + p_3$ (This time we used the statement)\\
Since x + 2 is even, one of the $ p_1,\ p_2,\  p_3$ must be 2.\\
Let's say $p_1 = 2$, then x + 2 = $ 2 + p_2 + p_3$\\
Thus x = $p_2 + p_3$ (sum of two primes) $(x > 2)$\\
We reached to Goldbach's Conjecture\\
\\
In conclusion, we can say that the statement and Goldbach's Conjecture are equivalent because we can reach the selecled one by using the another one.\\ 



\end{document}
